\chapter{Технологический раздел}

\section{Выбор программных средств реализации}

Для реализации программного обеспечения был выбран язык программирования C++. Данный язык был выбран по следующим причинам:
\begin{itemize}[label=---]
	\item в нём используется объектно--ориентированная парадигма;
	\item наличие стандартной библиотеки с готовыми контейнерами и алгоритмами над ними;
	\item наличие строгой типизации, что упрощает процесс отладки;
	\item наличие необходимых библиотек для реализации графического интерфейса.
\end{itemize}

Для реализации графического интерфейса был выбран фреймворк Qt, по следующим причинам:
\begin{itemize}[label=---]
	\item поддержка С++;
	\item наличие ПО для создания графических окон;
	\item широкая библиотека элементов для графического интерфейса и возможность их модернизации.
\end{itemize}

В качестве среды разработки был выбран Clion, по следующим причинам:
\begin{itemize}[label=---]
	\item встроенная поддержка средств отладки, анализа кода и сборки;
	\item поддержка для Qt;
	\item наличии студенческой лицензии.
\end{itemize}

\section{Описание пользовательского интерфейса интерфейса}

Реализованное ПО получает на вход характеристики многофункционального центра обслуживания, а на выходе выдаёт данные о работе элементов системы. На вход программе поступают информация о группах посетителей, о ресепшен и о группах окон обслуживания. Также пользователь вводит количество групп пользователей и окон и время моделирования.

Для \textbf{групп посетителей} вводятся следующие данные: 
\begin{itemize}[label=---]
	\item закон распределения интервала прихода посетителя;
	\item разброс интервала прихода;
	\item закон распределения интервала обработки;
	\item разброс интервала обработки.
\end{itemize}

Для \textbf{ресепшен} вводятся следующие данные: 
\begin{itemize}[label=---]
	\item закон распределения интервала обработки;
	\item разброс интервала обработки;
	\item максимальный размер очереди, начиная с которого посетители будут покидать систему без обслуживания.
\end{itemize}

Для \textbf{групп окон} вводятся следующие данные: 
\begin{itemize}[label=---]
	\item количество окон в группе;
	\item максимальный размер очереди, начиная с которого посетители будут покидать систему без обслуживания;
	\item группы посетителей, которые обрабатывает данные окна.
\end{itemize}

После моделирования собираются данные по интересующим переходам и позициям, они обрабатываются и выводятся на экран. Выводится информация о работе все системы, а также каждой очереди, ресепшена и групп окон.

Для \textbf{системы} выводится:
\begin{itemize}[label=---]
	\item количество пришедших клиентов;
	\item процент успешно обслуженных клиентов;
	\item процент клиентов, ушедших из системы без обслуживания, из--за большой очереди.
\end{itemize}
Сумма успешно обслуженных клиентов и ушедших может быть не равна количеству пришедших, так как на момент конца моделирования в системе остались ещё не вышедшие заявки.

Для \textbf{очередей} выводится:
\begin{itemize}[label=---]
	\item название;
	\item максимальный размер;
	\item количество попыток входа (удачные и неудачные);
	\item количество входов в пустую очередь;
	\item средний размер очереди;
	\item средний время пребывания клиента в очереди;
	\item количество клиентов, которые не смогли встать в очередь из--за её большого размера.
\end{itemize}

Для \textbf{обработчиков} выводится:
\begin{itemize}[label=---]
	\item название;
	\item количество успешно обработнных клиентов;
	\item среднее относительное время работы окон;
	\item среднее время работы окон.
\end{itemize}

\section{Пример работы программы}

На рисунке~\ref{img:in} подставлен интерфейс разработанной программы, в котором можно ввести параметры центра обслуживания описанные выше.

\imgScale{0.7}{in}{Ввод параметров многофункционального центра обслуживания}
\FloatBarrier

На рисунке~\ref{img:out} подставлен результат моделирования, смысл значений которого были описаны выше, для параметров изображённых на рисунке~\ref{img:in}.

\imgScale{0.6}{out}{Результаты моделирования}
\FloatBarrier

\section{Реализация программного обеспечения}

Разработанное ПО состоит из следующих логических частей:
\begin{itemize}[label=---]
	\item сеть Петри: построение, функционирование и сбор статистических данных;
	\item моделирующий алгоритм: управление сетью Петри;
	\item графический интерфейс.
\end{itemize}

\subsection{Реализация сети Петри}

В листингах~\ref{lst:find}--\ref{lst:fire} приведена реализация основного функционала сети Петри, используемого в моделирующем алгоритме.

В листинге~\ref{lst:find}, представлен код функции поиска активных переходов. Перебираются все переходы, кроме уже запланированных на выполнение, и проверяются. В результирующий массив попадает индекс перехода и длительность срабатывания перехода. 

\begin{center}
	\captionsetup{justification=raggedright,singlelinecheck=off}
	\begin{lstlisting}[label=lst:find,caption=Функция поиска активного перехода ,showstringspaces=false]
vector<PetriEvent> PetriNet::find_fired_t_init() {
	vector<PetriEvent> res;
	
	for (int i = 0; i < t_num; ++i) {
		if (is_wait[i]) continue;
		
		auto fire_res = is_t_fire(i);
		if (fire_res.first) {
			if (fire_res.second > 0) is_wait[i] = true;
			res.push_back({i, fire_res.second});
		}
	}
	
	return res;
}
	\end{lstlisting}
\end{center}
\FloatBarrier

В листинге~\ref{lst:check}, представлен код функций проверки активности перехода. В зависимости от функционала перехода, выбирается функция проверки. Для переходов, которые пропускают только определённый тип фишек, проверятся наличие таковых в позиции. Для переходов, которые отражают работу окна, проверяется наличие фишек в соответствующей позиции и генерируется длительность срабатывания перехода, в зависимости от типа фишки. Для обычных переходов, проверяются все связанные дугами, в том числе ингибиторными, позиции на наличие необходимого числа фишек.

\begin{center}
	\captionsetup{justification=raggedright,singlelinecheck=off}
	\begin{lstlisting}[label=lst:check,caption=Функции проверки автивности перехода ,showstringspaces=false]
pair<bool, double> PetriNet::is_t_fire(int t_i) {
	if (selector_t.contains(t_i)) return check_selector_t(t_i);
	else if (win_poc.contains(t_i)) return check_win_proc_t(t_i);
	return check_usual_t(t_i);
}

pair<bool, double> PetriNet::check_selector_t(int t_i) {
	auto eff = cpn_t_effect[t_i];
	auto tokens = m[eff.pop_p];
	auto need_tokens = selector_t[t_i];
	
	for (auto &it: tokens) {
		if (need_tokens.contains(it)) return {true, timing[t_i]->gen()};
	}
	
	return {false, 0};
}

pair<bool, double> PetriNet::check_win_proc_t(int t_i) {
	auto eff = cpn_t_effect[t_i];
	auto tokens = m[eff.pop_p];
	
	if (tokens.empty()) return {false, 0};
	
	auto type = tokens.front();
	return {true, type_proc_distro[type]->gen()};
}

pair<bool, double> PetriNet::check_usual_t(int t_i) {
	for (auto const &it: t_check[t_i])
		if (m[it.p_index].size() < it.min_num || !m[it.p_index].empty() && it.is_ing)
			return {false, 0};
	
	return {true, timing[t_i]->gen()};
}
	\end{lstlisting}
\end{center}
\FloatBarrier

В листинге~\ref{lst:fire}, представлен код функций срабатывания перехода. В зависимости от функционала перехода, выбирается функция срабатывания. Для всех переходов снимается флаг планирования, чтоб их проверяли на активность. Для генераторов, добавляются фишки соответствующего типа в позицию. Для переходов, которые пропускают только определённый тип фишек, происходит выбор фишки нужного типа и перенос её в выходную позицию. Для обычных переходов, переносится верхняя фишка из очереди.

\begin{center}
	\captionsetup{justification=raggedright,singlelinecheck=off}
	\begin{lstlisting}[label=lst:fire,caption=Функции срабатывания перехода ,showstringspaces=false]
void PetriNet::fire_t(PetriEvent event) {
	t_stats[event.t_i].fire_num++;
	
	int t_i = event.t_i;
	is_wait[t_i] = false;
	if (gen_t.contains(t_i)) {
		process_gen_t(event);
	} else if (selector_t.contains(t_i)) {
		process_selector_t(event);
	} else {
		process_usual_t(event);
	}
}

void PetriNet::process_gen_t(PetriEvent event) {
	int t_i = event.t_i;
	auto eff = cpn_t_effect[t_i];
	
	push_p_stats(eff.push_p);
	m[eff.push_p].push_back(gen_t[t_i]);
	logs.push_back({t_i, eff.pop_p, eff.push_p, gen_t[t_i], event.gen_time, event.sys_time});
}

void PetriNet::process_selector_t(PetriEvent event) {
	int t_i = event.t_i;
	auto eff = cpn_t_effect[t_i];
	auto &tokens = m[eff.pop_p];
	auto need_tokens = selector_t[t_i];
	
	int val = -1;
	for (auto it = begin(tokens); it != end(tokens); ++it) {
		if (need_tokens.contains(*it)) {
			val = *it;
			tokens.erase(it);
			break;
		}
	}
	
	if (eff.add_p >= 0) {
		if (eff.add_val > 0) {
			push_p_stats(eff.add_p);
			m[eff.add_p].push_back(-1);
			logs.push_back({t_i, -1, eff.add_p, -1, event.gen_time, event.sys_time});
		} else {
			m[eff.add_p].pop_front();
			logs.push_back({t_i, eff.add_p, -1, -1, event.gen_time, event.sys_time});
		}
	}
	
	int to = -1;
	if (eff.push_p >= 0) {
		push_p_stats(eff.push_p);
		m[eff.push_p].push_back(val);
		to = eff.push_p;
	}
	logs.push_back({t_i, eff.pop_p, to, val, event.gen_time, event.sys_time});
}

void PetriNet::process_usual_t(PetriEvent event) {
	int t_i = event.t_i;
	auto eff = cpn_t_effect[t_i];
	
	if (eff.add_p >= 0) {
		if (eff.add_val > 0) {
			push_p_stats(eff.add_p);
			m[eff.add_p].push_back(-1);
			logs.push_back({t_i, -1, eff.add_p, -1, event.gen_time, event.sys_time});
		} else {
			m[eff.add_p].pop_front();
			logs.push_back({t_i, eff.add_p, -1, -1, event.gen_time, event.sys_time});
		}
	}
	
	int val = m[eff.pop_p].front();
	m[eff.pop_p].pop_front();
	int to = -1;
	if (eff.push_p >= 0) {
		push_p_stats(eff.push_p);
		m[eff.push_p].push_back(val);
		to = eff.push_p;
	}
	logs.push_back({t_i, eff.pop_p, to, val, event.gen_time, event.sys_time});
}

	\end{lstlisting}
\end{center}
\FloatBarrier

\subsection{Реализация моделирующего алгоритма}

В листинге~\ref{lst:delft} приведён класс представляющий моделирующий алгоритм.

\begin{center}
	\captionsetup{justification=raggedright,singlelinecheck=off}
	\begin{lstlisting}[label=lst:delft,caption=Класс моделирующего алгоритма ,showstringspaces=false]
struct DelftParam {
	int step;
	int mult;
	int m_time;
};

class Delft {
	private:
	DelftParam param;
	
	vector<queue<PetriEvent>> low_list;
	int low_pointer;
	int low_period;
	
	vector<vector<int>> top_lists;
	vector<int> level_pointer;
	vector<int> level_period;
	
	shared_ptr<PetriNet> net;
	public:
	Delft(DelftParam param, shared_ptr<PetriNet> net);
	
	void run();
	
	private:
	int execute(int level, int &event_num);
	
	int execute_zero_lvl(int &event_num);
	
	int execute_events();
	
	void insert_events(const vector<PetriEvent> &events, double now);
};		
	\end{lstlisting}
\end{center}
\FloatBarrier

В листинге~\ref{lst:run} приведён код функции, которая обрабатывает самый верхний список часовой структуры. Сначала вставляются события, готовые к выполнению с самого начала. Далее в цикле проходятся по ячейкам списка, пока не найдут не пустую, значение ячейки говорит о количестве событий запланированных в этом промежутке времени. При наличии событий, вызывается функция обработки нижележащего списка и инкрементируется указатель списка.

\begin{center}
	\captionsetup{justification=raggedright,singlelinecheck=off}
	\begin{lstlisting}[label=lst:run,caption=Функция обработки верхнего списка,showstringspaces=false]
void Delft::run() {
	insert_events(net->find_fired_t_init(), 0);
	
	int level = top_lists.size() - 1;
	while (true) {
		while (level_pointer[level] < top_lists[level].size() && top_lists[level][level_pointer[level]] == 0)
		level_pointer[level]++;
		if (level_pointer[level] == top_lists[level].size() && top_lists[level][level_pointer[level] - 1] == 0)
		break;
		
		if (level - 1 < 0) {
			low_pointer = level_pointer[level] * param.mult;
			execute_zero_lvl(top_lists[level][level_pointer[level]]);
		} else {
			level_pointer[level - 1] = level_pointer[level] * param.mult;
			execute(level - 1, top_lists[level][level_pointer[level]]);
		}
		
		level_pointer[level]++;
	}
}	
	\end{lstlisting}
\end{center}
\FloatBarrier

В листинге~\ref{lst:execute} приведён код функции, которая обрабатывает промежуточные списки часовой структуры. Работает аналогично функции обработки верхнего уровня, только дополнительно декрементирует счётчик событий ячейки уровнем выше и возвращает количество обработанных событий.

\begin{center}
	\captionsetup{justification=raggedright,singlelinecheck=off}
	\begin{lstlisting}[label=lst:execute,caption=Функция обработки промежуточных списков,showstringspaces=false]
int Delft::execute(int level, int &event_num) {
	int event_done = 0;
	
	while (event_num > 0) {
		while (level_pointer[level] < top_lists[level].size() && top_lists[level][level_pointer[level]] == 0)
		level_pointer[level]++;
		
		int exc = 0;
		if (level - 1 < 0) {
			low_pointer = level_pointer[level] * param.mult;
			exc = execute_zero_lvl(top_lists[level][level_pointer[level]]);
		} else {
			level_pointer[level - 1] = level_pointer[level] * param.mult;
			exc = execute(level - 1, top_lists[level][level_pointer[level]]);
		}
		
		event_num -= exc;
		event_done += exc;
		
		level_pointer[level]++;
	}
	
	return event_done;
}		
	\end{lstlisting}
\end{center}
\FloatBarrier

В листинге~\ref{lst:zero} приведён код функции, которая обрабатывает список нулевого уровня часовой структуры. Работает аналогично функции обработки промежуточных уровней уровня, только вместо обработки нижележащих уровней начинает выполнение событий.

\begin{center}
	\captionsetup{justification=raggedright,singlelinecheck=off}
	\begin{lstlisting}[label=lst:zero,caption=Функция обработки нижнего списка,showstringspaces=false]
int Delft::execute_zero_lvl(int &event_num) {
	int event_done = 0;
	
	while (event_num > 0) {
		while (low_pointer < low_list.size() && low_list[low_pointer].empty())
		low_pointer++;
		
		int exc = execute_events();
		
		event_num -= exc;
		event_done += exc;
		
		low_pointer++;
	}
	
	return event_done;
}	
	\end{lstlisting}
\end{center}
\FloatBarrier

В листинге~\ref{lst:event} приведён код функции, выполняет события из ячейки списка нулевого уровня. Пока список событий не пуст выполняется все события лежащие в нём, а именно активируются переходы. Перед выполнением дополнительно проверяется возможность активации перехода, из--за конфликтов в сети. После выполнения снова происходит поиск активных переходов и вставка их в часовую структуру, при это новое событие может попасть прямо в обрабатываемую очередь.

\begin{center}
	\captionsetup{justification=raggedright,singlelinecheck=off}
	\begin{lstlisting}[label=lst:event,caption=Функция выполнения событий,showstringspaces=false]
int Delft::execute_events() {
	int event_done = 0;
	auto &q = low_list[low_pointer];
	
	while (!q.empty()) {
		event_done++;
		
		auto event = q.front();
		q.pop();
		auto check_res = net->is_t_fire(event.t_i);
		if (!check_res.first) continue;
		
		net->fire_t(event);
		insert_events(net->find_fired_t_init(), event.sys_time);
	}
	
	return event_done;
}
	\end{lstlisting}
\end{center}
\FloatBarrier


В листинге~\ref{lst:insert} приведён код функции вставки событий в часовую структуру. Для события последовательно вычисляются индексы в верхних списках и инкрементируется счётчики в соответствующих ячейках. Для ячейки нижнего уровня происходит вставка в ей список событий. Если событие должно выполниться моментально, то просто происходит вставка по текущим индексам.

\begin{center}
	\captionsetup{justification=raggedright,singlelinecheck=off}
	\begin{lstlisting}[label=lst:insert,caption=Функция выполнения событий,showstringspaces=false]
void Delft::insert_events(const vector<PetriEvent> &events, double now) {
	for (auto it: events) {
		if (it.gen_time == 0) {
			it.sys_time = now;
			for (int i = 0; i < top_lists.size(); ++i) top_lists[i][level_pointer[i]]++;
			low_list[low_pointer].push(it);
		} else {
			double time = now + it.gen_time;
			if (time > param.m_time)continue;
			
			it.sys_time = time;
			time -= EPS;
			
			for (int i = 0; i < top_lists.size(); ++i) {
				int index = int(time / level_period[i]);
				top_lists[i][index]++;
			}
			
			low_list[int(time / low_period)].push(it);
		}
	}
}
	\end{lstlisting}
\end{center}
\FloatBarrier

\section*{Вывод}

В данном разделе был приведён выбор средств программной реализации. Описан интерфейс программы, а также входные и выходные данные. Представлена и описана основная часть реализации сетей Петри и моделирующего алгоритма.
