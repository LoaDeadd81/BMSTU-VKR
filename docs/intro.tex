\chapter*{ВВЕДЕНИЕ}
\addcontentsline{toc}{chapter}{ВВЕДЕНИЕ}

Моделирование --- процесс замещения одного объекта другим с целью получения информации о важнейших свойствах объекта--оригинала с помощью объекта--модели. Широко применяется в научных исследования и в прикладных задачах в различных областях. Компьютерное моделирование является одним из эффективных методов изучения сложных систем. Модели проще и удобнее исследовать, когда реальные эксперименты затруднены из--за финансовых или физических препятствий. Формализованность позволяет чётко обозначить основные факторы, определяющие свойства изучаемого объекта--оригинала и связи между ними.

В современном мире управление операционной деятельностью и оптимизация процессов обслуживания стали ключевыми вопросами для организаций, предоставляющих различные виды услуг. Одной из важных составляющих этой области является моделирование многофункциональных центров обслуживания, пересекающихся с различными видами услуг и комплексными процессами. Классификация методов моделирования многофункциональных центров обслуживания и анализ их применимости представляют высокую актуальность для исследователей и практиков.

Моделирование многофункциональных центров позволяет проанализировать и проконтролировать правильность функционирования систем, без больших затрат на оборудование, персонал и обслуживание. Также даёт возможность обнаружить ошибки проектирования на этапе подготовки, а не во время эксплуатации, что также значительно снижает расходы.

Актуальность этой темы объясняется большим и постоянно растущем спросом людей на услуги предоставляемые данными центрами. Только за 2022 в Москве в таких центрах было оказано более 27 миллионов услуг~\cite{actual}. Для анализа и контроля правильности функционирования многофункциональных центров обслуживания применяется моделирование. Это позволяет сокращать время проектирования, уменьшать конечную стоимость создания центров, исключаем множественные исправления дефектов выявленных в ходе эксплуатации.

Целью данной научно--исследовательской работы --- провести обзор и сравнить существующие методы моделирования многофункциональных центров обслуживания.

Для достижения поставленной в работе цели предстоит решить следующие задачи:
\begin{itemize}[label=---]
	\item изучить основные понятия моделирования многофункциональных центров обслуживания;
	\item описать и классифицировать существующие методы;
	\item произвести сравнительный анализ рассмотренных методов.
\end{itemize}
