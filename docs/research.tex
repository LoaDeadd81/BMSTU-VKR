\chapter{Исследовательский раздел}
	
\section{Исследование времени работы ПО}

В качестве модели взят многофункциональный центр с тремя генераторами и тремя группами окон по три окна каждая. Каждая группа окон обрабатывает только один тип заявок. Временные характеристики типов заявок идентичны. В моделирующем алгоритме помимо времени моделирования ещё есть параметры элементарного шага и множитель уровней, которые почти не влияют на время работы алгоритма, так как большую часть времени занимает функция поиска активных переходов.
На рисунке~\ref{img:ro} представлена результаты замеров времени работы алгоритма в зависимости от загрузки. Время моделирования 2000. Замеры проводились по 10 раз и бралось среднее.

\imgScale{0.5}{ro}{Зависимость времени работы от загрузки }
\FloatBarrier

Из результатов видно, что зависимость имеет линейный характер. Из графика видно, что когда в систему приходит больше заявок чем обрабатывается, то график растёт значительно медленнее, так как из сети начинают уходить фишки раньше, то есть теперь некоторые фишки проходят меньшее количество переходов, что уменьшает общий объём работы.

На рисунке~\ref{img:m_time} изображена зависимость времени работы от времени моделирования для различных загрузок. Модель аналогична прошлой.  

\imgScale{0.5}{m_time}{Зависимость времени работы от времени моделирования}
\FloatBarrier

Из результатов видно что зависимость имеет линейный характер. Также чем больше нагрузку тем быстрее растёт время. При загрузке больше 1, графики близки к графику с загрузкой равной 1.

\section{Сравнений с GPSS World}

Используется модель такая же как и ранее. В листинге~\ref{lst:gpss} приведено описание аналогичной модели используемой в GPSS World c загрузкой 0.5 и временем моделирования 2000.

\begin{center}
	\captionsetup{justification=raggedright,singlelinecheck=off}
	\begin{lstlisting}[label=lst:gpss,caption=Модель GPSS World,showstringspaces=false]
SIMULATE
wg_1	STORAGE 3
wg_2	STORAGE 3
wg_3	STORAGE 3

GENERATE 30,0.1
ASSIGN type,1
TRANSFER ,recep

GENERATE 30,0.1
ASSIGN type,2
TRANSFER ,recep

GENERATE 30,0.1
ASSIGN type,3
TRANSFER ,recep


recep	TEST L q$reception_q,10,out_q
QUEUE reception_q
SEIZE reception
DEPART reception_q
ADVANCE 5,0.1
RELEASE reception

TEST L P$type,3,wg_3_p
TEST L P$type,2,wg_2_p


wg_1_p	TEST L q$wg_1_q,10,out_wg_1
QUEUE wg_1_q
ENTER wg_1
DEPART wg_1_q
ADVANCE 45,0.1
LEAVE wg_1
TRANSFER ,out

wg_2_p	TEST L q$wg_2_q,10,out_wg_2
QUEUE wg_2_q
ENTER wg_2
DEPART wg_2_q
ADVANCE 45,0.1
LEAVE wg_2
TRANSFER ,out

wg_3_p	TEST L q$wg_3_q,10,out_wg_3
QUEUE wg_3_q
ENTER wg_3
DEPART wg_3_q
ADVANCE 45,0.1
LEAVE wg_3
TRANSFER ,out


out	TERMINATE
out_q	TERMINATE
out_wg_1	TERMINATE
out_wg_2	TERMINATE
out_wg_3	TERMINATE


GENERATE 2000
TERMINATE 1
START 1
	\end{lstlisting}
\end{center}
\FloatBarrier

На рисунке~\ref{img:cmp_l} изображён график зависимости времени работы от загрузки для реализованного ПО и GPSS World.

\imgScale{0.6}{cmp_l}{Сравнение зависимостей времени работы от загрузки}
\FloatBarrier

На рисунке~\ref{img:cmp_t} изображён график зависимости времени работы от времени моделирования для реализованного ПО и GPSS World.

\imgScale{0.6}{cmp_t}{Сравнение зависимостей времени работы от времени моделирования}
\FloatBarrier

Можно сделать вывод от том, что обе зависимости имею линейный характер, как у реализованного ПО так и GPSS World, но график реализованного ПО растёт быстрее. Также GPSS World в среднем работает в 5 раз быстрее. Но ПО реализованное в рамках данной работы обладает более удобным графическим интерфейсом, так как пользователю не нужно изучать специальный язык для создания моделей, а затем описывать модель в текстовом файле, достаточно заполнить форму на главном экране. Также результаты представлены в виде удобных для чтения таблиц, а не в текстовом формате. Помимо этого реализованное ПО способно визуализировать работу модели.

\section{Вывод}

Можно сделать вывод, что для рассмотренной модели параметры моделирующего алгоритма почти не оказывают значения на время работы, так как большую часть времени работы занимает функция поиска активных переходов. Также зависимость времени работы реализованного ПО как от загрузки, так и от времени моделирования имеет линейный характер. Также если в систему начинает поступать больше заявок, чем она может обработать, то зависимость времени работы от загрузки, становится  более пологой, потому что из систему начинают уходить заявки. Фишки ассоциированные с этими заявками проходят меньшее количество переходов, то есть уменьшается общая работа системы.

По сравнению с GPSS World реализованное ПО работает в среднем в 5 раз медленнее и тоже имеет линейный характер, но растёт быстрее. Но реализованное ПО обладает графическим интерфейсом с формами ввода, результатами представленными в виде таблиц и возможностью визуализации работы модели.