\chapter{Исследовательский раздел}
	
\section{Исследование времени работы ПО}

В качестве модели взят многофункциональный центр с тремя генераторами и тремя группами окон по три окна каждая. Каждое окно обрабатывает только один тип заявок. Временные характеристики типов заявок идентичны. В моделирующем алгоритме помимо времени моделирования ещё есть параметры элементарного шага и множитель уровней.
На рисунка~\ref{img:m_2}--~\ref{img:m_10} представлены результаты замеров времени работы алгоритма в зависимости от загрузки, шага и множителя уровней. На каждом изображении нарисованы графики зависимости времени работы от загрузки, для различных шагов. Шаг варьируются в интервале от 1 до 5, так как минимальный интервал времени в системе равен 5. Если поставить больше, то события могут выполнятся в неверном порядке, так как внутри ячейки нижнего уровня они не сортируются. Для каждого рисунка используется разный множитель уровней при котором в часовой структуре 11, 5 и 3 уровня соответственно. Время моделирования 2000. Замеры проводились по 10 раз и бралось среднее.

\imgScale{0.4}{m_2}{Зависимость времени работы от загрузки при множителе равным 2}
\FloatBarrier

\imgScale{0.4}{m_6}{Зависимость времени работы от загрузки при множителе равным 6}
\FloatBarrier

\imgScale{0.4}{m_10}{Зависимость времени работы от загрузки при множителе равным 10}
\FloatBarrier

Из результатов видно, что множитель уровней не оказывать почти никакого влияния в рассмотренном случае. Также видно, что программа с шагом 5 работает быстрее всего, приморено на 1\% быстрее шага 2, самый медленный. Можно сделать вывод, что не стоит брать шаг равным 2, а также о линейном виде зависимости времени работы от загрузки системы.

На рисунке~\ref{img:m_time} изображена зависимость времени работы от времени моделирования для различных загрузок. Модель аналогична прошлой.  

\imgScale{0.5}{m_time}{Зависимость времени работы от времени моделирования}
\FloatBarrier

Из результатов видно что зависимость имеет линейный характер. Также чем больше нагрузку тем быстрее растёт время.

\section{Сравнений с GPSS World}

Используется модель такая же как и ранее. В листинге~\ref{lst:gpss} приведено описание аналогичной модели используемой в GPSS World c загрузкой 0.5 и временем моделирования 2000.

\begin{center}
	\captionsetup{justification=raggedright,singlelinecheck=off}
	\begin{lstlisting}[label=lst:gpss,caption=Модель GPSS World,showstringspaces=false]
SIMULATE
wg_1	STORAGE 3
wg_2	STORAGE 3
wg_3	STORAGE 3

GENERATE 30,0.1
ASSIGN type,1
TRANSFER ,recep

GENERATE 30,0.1
ASSIGN type,2
TRANSFER ,recep

GENERATE 30,0.1
ASSIGN type,3
TRANSFER ,recep


recep	TEST L q$reception_q,10,out_q
QUEUE reception_q
SEIZE reception
DEPART reception_q
ADVANCE 5,0.1
RELEASE reception

TEST L P$type,3,wg_3_p
TEST L P$type,2,wg_2_p


wg_1_p	TEST L q$wg_1_q,10,out_wg_1
QUEUE wg_1_q
ENTER wg_1
DEPART wg_1_q
ADVANCE 45,0.1
LEAVE wg_1
TRANSFER ,out

wg_2_p	TEST L q$wg_2_q,10,out_wg_2
QUEUE wg_2_q
ENTER wg_2
DEPART wg_2_q
ADVANCE 45,0.1
LEAVE wg_2
TRANSFER ,out

wg_3_p	TEST L q$wg_3_q,10,out_wg_3
QUEUE wg_3_q
ENTER wg_3
DEPART wg_3_q
ADVANCE 45,0.1
LEAVE wg_3
TRANSFER ,out


out	TERMINATE
out_q	TERMINATE
out_wg_1	TERMINATE
out_wg_2	TERMINATE
out_wg_3	TERMINATE


GENERATE 2000
TERMINATE 1
START 1
	\end{lstlisting}
\end{center}
\FloatBarrier

На рисунке~\ref{img:cmp_l} изображён график зависимости времени работы от загрузки для реализованного ПО и GPSS World.

\imgScale{0.6}{cmp_l}{Сравнение зависимостей времени работы от загрузки}
\FloatBarrier

На рисунке~\ref{img:cmp_t} изображён график зависимости времени работы от времени моделирования для реализованного ПО и GPSS World.

\imgScale{0.6}{cmp_t}{Сравнение зависимостей времени работы от времени моделирования}
\FloatBarrier

Можно сделать вывод от том, что обе зависимости имею линейный характер, как у реализованного ПО так и GPSS World, но график реализованного ПО растёт быстрее. Также GPSS World в среднем работает в 5 раз быстрее.

\section{Вывод}

Можно сделать вывод, что для рассмотренной модели параметра моделирующего алгоритма почти не оказывают значения на время работы. Также зависимость времени работы реализованного ПО как от загрузки, так и от времени моделирования имеет линейный характер.

По сравнению с GPSS World реализованное ПО работает в среднем в 5 раз медленнее.