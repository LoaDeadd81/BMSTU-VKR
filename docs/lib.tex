\addcontentsline{toc}{chapter}{СПИСОК ИСПОЛЬЗОВАННЫХ ИСТОЧНИКОВ}

\renewcommand\bibname{СПИСОК ИСПОЛЬЗОВАННЫХ ИСТОЧНИКОВ}
\begin{thebibliography}{}
	
\setlength\bibindent{1.25cm}
\makeatletter
\let\old@biblabel\@biblabel
\def\@biblabel#1{\kern\bibindent\old@biblabel{#1}}
\makeatother

\bibitem{m_types} Градов В.М. Компьютерное моделирование / В.М. Градов, Г.В. Овечкин, П.В. Овечкин, И.В. Рудаков --- М.:КУРС ИНФРА--М, 2019. --- 264 С. 

\bibitem{actual} Посещаемость <<Мои документы>> в 2022 году [Электронный ресурс]. --- Режим доступа, URL: \url{https://www.mos.ru/news/item/117681073/} (дата обращения: 12.11.2023).

\bibitem{serv_types}  Расширение концепции ООО--модели для систем массового обслуживания на примере многофункционального центра предоставления государственных и муниципальных услуг / А.В. Чуев, С.А. Юдицкий, В.З. Магергут // Экономика. Информатика. --- 2015. --- №. 1. – С. 85--93.

\bibitem{har1} Пронникова Т.Ю. Применение имитационного моделирования для оптимизации бизнес-процессов обслуживания клиентов в многофункциональном центре / Т.Ю. Пронникова, М.Н. Рассказова // Прикладная математика и фундаментальная информатика. --- 2022. --- С. 122-123.

\bibitem{har2} Сутягина Н. И. Моделирование деятельности многофункционального центра как системы массового обслуживания // Карельский научный журнал. --- 2015. --- №. 1. --- С. 199--203.

\bibitem{ak_det} Моделирование систем / С.П. Бобков, Д.О. Бытев // ---Иваново:Изд. ИвГХТУ, 2008. --- 156 с. 

\bibitem{ak_types} Теория автоматов / Ожиганов А.А. // --- СПб.:НИУ ИТМО, 2013. --- 84 с. 

\bibitem{sheme_types} Моделирование систем / Альсова О.К. // --- Новосибирск:Изд-во НГТУ, 2007. --- 72 с.

\bibitem{va} Блюмин, С.Л. Дискретное моделирование систем автоматизации и управления / С.Л. Блюмин, А.М. Корнеев. --- Липецк:ЛЭГИ, 2005. ---\\124 с.

\bibitem{smo} Осипов Г.С. Математическое и имитационное моделирование систем массового
обслуживания / Г.С. Осипов --- М.: Издательский дом Академии Естествознания, 2017. --- 56 с.

\bibitem{smo_chan} Григорьева Т. Е., Донецкая А. А., Истигечева Е. В. Моделирование одноканальных и многоканальных систем массового обслуживания на примере билетной кассы автовокзала / Т.Е. Григорьева, А.А. Донецкая, Е.В. Истигечева  // Вестник Воронежского института высоких технологий. -- 2017. --- №. 1. --- С. 35--38.

\bibitem{petri} Мальков М.В. Сети Петри и моделирование / М.В. Мальков, С.Н. Малыгина // Труды Кольского научного центра РАН. --- 2010. --- №. 3. --- С. 35--40.

\bibitem{timed_petri} Мараховский В. Б. Моделирование параллельных процессов. Сети Петри. / В. Б. Мараховский, Л. Я. Розенблюм,  А. В. Яковлев --- СПб.:Профессиональная литература, 2014. --- 400 с.

\bibitem{cpnNoForm} Устимов К. О., Федоров Н. В. Автоматизация построения имитационной модели бизнес-процессов на основе методологии IDEF0 и раскрашенных сетей Петри // Горный информационно--аналитический бюллетень (научно-технический журнал). --- 2013. --- №. 12. --- С. 90--94.

\bibitem{cpn} Климов А. В. Цветные сети Петри и язык распределенного программирования UPL: их сравнение и перевод // Программные системы: теория и приложения. --- 2023. --- Т. 14. --- №. 4. --- С. 91-122.

\bibitem{time_alg} Кельтон В. Имитационное моделирование. Классика CS. 3-е изд. / В. Кельтон, А. Лоу --- СПб.:Издательская группа BHV, 2004. --- 847 с.

\bibitem{time_alg2} Советов Б. Я. Моделирование систем //  Б. Я. Советов, С. А. Яковлев --- М.: Высш. шк., 2001. --- 343 с.

\bibitem{delft} Аврамчук Е. Ф. Технология системного моделирования/ Е. Ф. Аврамчук, А. А. Вавилов, С. В. Емельянов и др.; Под общ. ред. С. В. Емельянова и др. — М.: Машиностроение, 1988. --- 520 с.

\end{thebibliography}